\chapter{Вычисление вероятностей}

\section*{Введение}

Вычисление вероятности события --- одна из основных задач в теории вероятностей. Обычно, эта задача формулируется в следующем виде: заданы события с известными вероятностями,
требуется вычислить вероятность некоторого другого события, которое с ними связано, то есть выражается через события с известными вероятностями с помощью операций объединения,
произведения, дополнения, разности.

Решение задач подобного типа состоит из двух частей: в первой части необходимо получить выражение для события с неизвестной вероятностью, во второй части
вычислить вероятность с использованием известных соотношений, к которым относятся:
\begin{enumerate}
    \item вероятность дополнительного события:
    \begin{equation}
        \probability{\overline{A}} = 1 - \probability{A} ,
    \end{equation}

    \item формула сложения:
    \begin{equation}
        \probability{A + B} = \probability{A} + \probability{B} - \probability{A B} ,
    \end{equation}

    \item формула умножения:
    \begin{equation}
        \probability{A \cdot B} = \probability{A} \cdot \conditionalprobability{B}{A},
    \end{equation}

    \item формула полной вероятности:
    \begin{equation}
        \probability{B} = \sum_{i=1}^n \probability{A_i} \cdot \conditionalprobability{B}{A_i} ,
    \end{equation}
    где $A_1$, \dots, $A_n$ --- полная группа событий.

    \item формула Байеса:
    \begin{equation}
        \conditionalprobability{A_i}{B}
        = \frac{\probability{A_iB}}{\probability{B}}
        = \frac{\probability{A_i} \cdot \conditionalprobability{B}{A_i}}{\sum_{i=1}^n \probability{A_i} \cdot \conditionalprobability{B}{A_i}} ,
    \end{equation}
    где $A_1$, \dots, $A_n$ --- полная группа событий.
\end{enumerate}

\section*{Задача 18.163}

Один раз подбрасывается игральный кубик. Заданы события:
\begin{itemize}
    \item $A = \event{\text{выпало простое число очков}}$ ,
    \item $B = \event{\text{выпало чётное число очков}}$ .
\end{itemize}
Вычислить условную вероятность $\conditionalprobability{A}{B}$.

\subsection*{Решение:}

Согласно определению условной вероятности:
\begin{equation}
    \conditionalprobability{A}{B} = \frac{\probability{A B}}{\probability{B}} .
\end{equation}

Вероятности $\probability{A B}$ и $\probability{B}$ вычислим, используя классическое определение вероятности. Элементарные исходы представляем число выпавших очков. При одном
подбрасывании кубика получим 6 элементарных исходов:
\begin{equation}
    \Omega = \set{1, 2, 3, 4, 5, 6}
\end{equation}
В событии $A B$ один элементарный исход:
\begin{equation}
    A B = \set { 2 } ,
\end{equation}
поэтому вероятность:
\begin{equation}
    \probability{A B} = \frac{\modulus{AB}}{\modulus{\Omega}} = \frac{1}{6} .
\end{equation}
В событии $B$ три элементарных исхода:
\begin{equation}
    B = \set { 2, 4, 6 } ,
\end{equation}
и вероятность:
\begin{equation}
    \probability{B} = \frac{\modulus{B}}{\modulus{\Omega}} = \frac{3}{6}.
\end{equation}
Таким образом, условная вероятность:
\begin{equation}
    \conditionalprobability{A}{B} = \frac{\frac{1}{6}}{\frac{3}{6}} = \frac{1}{3} .
\end{equation}

\subsection*{Ответ:}
$\frac{1}{3} .$

\section*{Задача 18.178}

Тетраэдр, три грани которого окрашены соответственно в красный, желтный и синий цвета, а четвертая грань содержит все три цвета, бросается наудачу. События:
\begin{itemize}
    \item $K = \event{\text{тетраэдр упал на грань, содержащую красный цвет}}$ ,
    \item $G = \event{\text{тетраэдр упал на грань, содержащую желтый цвет}}$ ,
    \item $S = \event{\text{тетраэдр упал на грань, содержащую синий цвет}}$ .
\end{itemize}

Показать, что указанные события попарно независимы, но не являются независимыми в совокупности.

\subsection*{Решение:}

Множество элементарных исходов содержит четыре элемента $\Omega = \set{\omega_1, \omega_2, \omega_3, \omega_4}$. Каждый исход соответствует грани, на которую падает тетраэдр:
\begin{itemize}
    \item $\omega_1$ --- грань красного цвета,
    \item $\omega_2$ --- грань желтого цвета,
    \item $\omega_3$ --- грань синего цвета,
    \item $\omega_4$ --- грань, содержащая все три цвета.
\end{itemize}

События:
\begin{gather}
    K = \set{\omega_1, \omega_4} , \\
    G = \set{\omega_2, \omega_4} , \\
    S = \set{\omega_3, \omega_4} .
\end{gather}
их вероятности:
\begin{gather}
    \probability{K} = \frac{\modulus{K}}{\modulus{\Omega}} = \frac{2}{4} = \frac{1}{2} , \\
    \probability{G} = \frac{\modulus{G}}{\modulus{\Omega}} = \frac{2}{4} = \frac{1}{2} , \\
    \probability{S} = \frac{\modulus{S}}{\modulus{\Omega}} = \frac{2}{4} = \frac{1}{2} .
\end{gather}

Попарные произведения событий:
\begin{gather}
    KG = \set{\omega_4} , \\
    KS = \set{\omega_4} , \\
    GS = \set{\omega_4} .
\end{gather}
и их вероятности:
\begin{gather}
    \probability{KG} = \frac{\modulus{KG}}{\modulus{\Omega}} = \frac{1}{4} , \\
    \probability{KS} = \frac{\modulus{KS}}{\modulus{\Omega}} = \frac{1}{4} , \\
    \probability{GS} = \frac{\modulus{GS}}{\modulus{\Omega}} = \frac{1}{4} .
\end{gather}

Проверяем условия попарной независимости --- вероятность произведения должна быть равна произведению вероятностей:
\begin{gather}
    \frac{1}{4} = \probability{KG} = \probability{K} \cdot \probability{G} = \frac{1}{2} \cdot \frac{1}{2} , \\
    \frac{1}{4} = \probability{KS} = \probability{K} \cdot \probability{S} = \frac{1}{2} \cdot \frac{1}{2} , \\
    \frac{1}{4} = \probability{GS} = \probability{G} \cdot \probability{S} = \frac{1}{2} \cdot \frac{1}{2} .
\end{gather}
Равенства выполняются --- все события попарно независимы.

Событие произведения всех событий:
\begin{equation}
    KGS = \set{\omega_4}
\end{equation}
и его вероятность:
\begin{equation}
    \probability{KGS} = \frac{\modulus{KGS}}{\modulus{\Omega}} = \frac{1}{4} .
\end{equation}

Проверяем условие независимости в совокупности:
\begin{equation}
    \frac{1}{4} = \probability{KGS} \neq \probability{K} \cdot \probability{G} \cdot \probability{S} = \frac{1}{2} \cdot \frac{1}{2} \cdot \frac{1}{2} = \frac{1}{8} .
\end{equation}
Равенство не выполняется --- события не являются независимыми в совокупности.

\section*{Задача 18.182}

В ящике лежат 12 красных, 8 зеленых и 10 синих шаров. Наудачу вынимают два шара. Найти вероятность того, что будут вынуты шары разного цвета, при условии, что не вынут синий шар.

\subsection*{Решение}

\subsubsection{Вариант I}

Нас интересуют два события: $A = \event{\text{извлечены шары разных цветов}}$ и $B = \event{\text{не вынут синий шар}}$. Как связаны между собой события $A$ и $B$?

Подумаем о том, какие события могут происходить в нашем эксперименте. Очевидно, что происходит только одно из следующих событий:
\begin{itemize}
    \item KK --- извлечены два красных шара,
    \item KС --- извлечены красный и синий шары,
    \item KЗ --- извлечены красный и зелёный шары,
    \item СС --- извлечены два синих шара,
    \item СЗ --- извлечены синий и зеленый шары,
    \item ЗЗ --- извлечены два зеленых шара.
\end{itemize}
В этих событиях порядок извлечения не учитывается, например, в событие КС попадают и элементарные исходы, в которых первый извлеченный шар красный, а второй синий,
и элементарные исходы, в которых первым оказался синий шар, а второй красный.

Легко видеть, что события $A$ и $B$ имеют следующее представление:
\begin{gather}
    A = \text{КС} + \text{КЗ} + \text{СЗ} , \\
    B = \text{KK} + \text{КЗ} + \text{ЗЗ} .
\end{gather}
Отсюда
\begin{equation}
    A B = \text{КЗ} ,
\end{equation}
а искомая условная вероятность
\begin{equation}
    \conditionalprobability{A}{B}
    = \frac{\probability{AB}}{\probability{B}}
    = \frac{\probability{\text{КЗ}}}{\probability{\text{КК} + \text{КЗ} + \text{ЗЗ}}} .
\end{equation}

Заметим, что события КК, КЗ и ЗЗ несовместны (не могут происходит одновременно, не содержат общих элементарных исходов), поэтому по аксиоме аддитивности меры
$\probability{\cdot}$:
\begin{equation}
    \probability{\text{КК} + \text{КЗ} + \text{ЗЗ}} = \probability{\text{КК}} + \probability{\text{КЗ}} + \probability{\text{ЗЗ}}.
\end{equation}

Таким образом,
\begin{equation}
    \conditionalprobability{A}{B}
    = \frac{\probability{\text{КЗ}}}{\probability{\text{КК}} + \probability{\text{КЗ}} + \probability{\text{ЗЗ}}}
\end{equation}
и осталось вычислить вероятности событий КК, КЗ и ЗЗ. Это нужно сделать путём определения множества всех элементарных исходов и определения их вероятностей.

Занумеруем все шары в ящике: красные шары имеют номера 1 -- 12, синие шары --- 13 -- 20, зелёные --- 21 -- 30. Элементарные исходы представляем в виде пары двух чисел,
соответствующих тем номерам шаров, которые были извлечены из ящика:
\begin{equation}
    \Omega = \set{\left ( i, j \right ): i, j \in \set{1, \dots, 30}, i \neq j}.
\end{equation}
При таком представлении все элементарные исходы являются равновероятными, это значит, что для вычисления вероятностей событий достаточно всего лишь посчитать количества
элементарных исходов в каждом событии. Всего элементарных исходов:
\begin{equation}
    \modulus{\Omega} = 30 \cdot 29.
\end{equation}
А в событиях:
\begin{align}
    \modulus{\text{КК}} = 12 \cdot 11 & \rightarrow \probability{\text{КК}} = \frac{\modulus{\text{КК}}}{\modulus{\Omega}} = \frac{12 \cdot 11}{30 \cdot 29} , \\
    \modulus{\text{КЗ}} = 12 \cdot 8 + 8 \cdot 12 & \rightarrow \probability{\text{КЗ}} = \frac{\modulus{\text{КЗ}}}{\modulus{\Omega}} = \frac{12 \cdot 8 + 8 \cdot 12}{30 \cdot 29} , \\
    \modulus{\text{ЗЗ}} = 8 \cdot 7 & \rightarrow \probability{\text{ЗЗ}} = \frac{\modulus{\text{ЗК}}}{\modulus{\Omega}} = \frac{8 \cdot 7}{30 \cdot 29} .
\end{align}

Искомая условная вероятность:
\begin{multline}
    \conditionalprobability{A}{B}
    = \frac{\frac{12 \cdot 8 + 8 \cdot 12}{30 \cdot 29}}{\frac{12 \cdot 11}{30 \cdot 29} + \frac{12 \cdot 8 + 8 \cdot 12}{30 \cdot 29} + \frac{8 \cdot 7}{30 \cdot 29}}
    = \frac{12 \cdot 8 + 8 \cdot 12}{12 \cdot 11 + 12 \cdot 8 + 8 \cdot 12 + 8 \cdot 7} = \\
    %
    = \frac{3 \cdot 8 + 8 \cdot 3}{3 \cdot 11 + 3 \cdot 8 + 8 \cdot 3 + 2 \cdot 7}
    = \frac{48}{33 + 48 + 14}
    = \frac{48}{95} .
\end{multline}

\subsubsection*{Вариант II}

Сократим множество всех элементарных исходов до множества $B$:
\begin{equation}
    \Omega \rightarrow \Omega^\prime = B .
\end{equation}
Сокращенное множество элементарных исходов $\Omega^\prime$ по-прежнему представляется множеством пар:
\begin{equation}
    \Omega^\prime = \set{\left ( i, j \right ): i, j \in \set{1, \dots, 12, 13, \dots, 20}} ,
\end{equation}
но в нём нет номеров, соответствующих шарам синего цвета. Общее количество элементарных исходов:
\begin{equation}
    \modulus{\Omega^\prime} = 20 \cdot 19 .
\end{equation}

Рассматривается событие $A^\prime = \set{\text{извлечены шары разных цветов}}$. Оно образовано парами, в которых один номер соответствует шару красного цвета, а другой - зеленого
(событие $A^\prime$ является аналогом события $AB$ из варианта I):
\begin{equation}
    A^\prime = \set{\left ( i, j \right ): i \in \set{1, \dots, 12}, j \in \set{13, \dots, 20}} + \set{\left ( i, j \right ): i \in \set{13, \dots, 20}, j \in \set{1, \dots, 12}} .
\end{equation}
Количество элементарных исходов:
\begin{equation}
    \modulus{A^\prime} = 12 \cdot 8 + 8 \cdot 12 .
\end{equation}
Таким образом, вероятность
\begin{equation}
    \probability{A^\prime}
    = \frac{\modulus{A^\prime}}{\modulus{\Omega^\prime}}
    = \frac{12 \cdot 8 + 8 \cdot 12}{20 \cdot 19}
    = \frac{3 \cdot 8 + 8 \cdot 3}{5 \cdot 19}
    = \frac{48}{95} .
\end{equation}

\subsection*{Ответ:}
$\frac{48}{95}$

% 167 179