\chapter{Вычисление вероятностей}

\section*{Введение}

Вычисление вероятности события --- одна из основных задач в теории вероятностей. Обычно, эта задача формулируется в следующем виде: заданы события с известными вероятностями,
требуется вычислить вероятность некоторого другого события, которое с ними связано, то есть выражается через события с известными вероятностями с помощью операций объединения,
произведения, дополнения, разности.

Решение задач подобного типа состоит из двух частей: в первой части необходимо получить выражение для события с неизвестной вероятностью, во второй части
вычислить вероятность с использованием известных соотношений, к которым относятся:
\begin{enumerate}
    \item вероятность дополнительного события:
    \begin{equation}
        \probability{\overline{A}} = 1 - \probability{A} ,
    \end{equation}

    \item формула сложения:
    \begin{equation}
        \probability{A + B} = \probability{A} + \probability{B} - \probability{A B} ,
    \end{equation}

    \item формула умножения:
    \begin{equation}
        \probability{A \cdot B} = \probability{A} \cdot \conditionalprobability{B}{A},
    \end{equation}

    \item формула полной вероятности:
    \begin{equation}
        \probability{B} = \sum_{i=1}^n \probability{A_i} \cdot \conditionalprobability{B}{A_i} ,
    \end{equation}
    где $A_1$, \dots, $A_n$ --- полная группа событий.

    \item формула Байеса:
    \begin{equation}
        \conditionalprobability{A_i}{B}
        = \frac{\probability{A_iB}}{\probability{B}}
        = \frac{\probability{A_i} \cdot \conditionalprobability{B}{A_i}}{\sum_{i=1}^n \probability{A_i} \cdot \conditionalprobability{B}{A_i}} ,
    \end{equation}
    где $A_1$, \dots, $A_n$ --- полная группа событий.
\end{enumerate}

\section*{Задача 18.163}

Один раз подбрасывается игральный кубик. Заданы события:
\begin{itemize}
    \item $A = \event{\text{выпало простое число очков}}$ ,
    \item $B = \event{\text{выпало чётное число очков}}$ .
\end{itemize}
Вычислить условную вероятность $\conditionalprobability{A}{B}$.

\subsection*{Решение:}

Согласно определению условной вероятности:
\begin{equation}
    \conditionalprobability{A}{B} = \frac{\probability{A B}}{\probability{B}} .
\end{equation}

Вероятности $\probability{A B}$ и $\probability{B}$ вычислим, используя классическое определение вероятности. Элементарные исходы представляем число выпавших очков. При одном
подбрасывании кубика получим 6 элементарных исходов:
\begin{equation}
    \Omega = \set{1, 2, 3, 4, 5, 6}
\end{equation}
В событии $A B$ один элементарный исход:
\begin{equation}
    A B = \set { 2 } ,
\end{equation}
поэтому вероятность:
\begin{equation}
    \probability{A B} = \frac{\modulus{AB}}{\modulus{\Omega}} = \frac{1}{6} .
\end{equation}
В событии $B$ три элементарных исхода:
\begin{equation}
    B = \set { 2, 4, 6 } ,
\end{equation}
и вероятность:
\begin{equation}
    \probability{B} = \frac{\modulus{B}}{\modulus{\Omega}} = \frac{3}{6}.
\end{equation}
Таким образом, условная вероятность:
\begin{equation}
    \conditionalprobability{A}{B} = \frac{\frac{1}{6}}{\frac{3}{6}} = \frac{1}{3} .
\end{equation}

\subsection*{Ответ:}
$\frac{1}{3} .$

\section*{Задача 18.178}

Тетраэдр, три грани которого окрашены соответственно в красный, желтный и синий цвета, а четвертая грань содержит все три цвета, бросается наудачу. События:
\begin{itemize}
    \item $K = \event{\text{тетраэдр упал на грань, содержащую красный цвет}}$ ,
    \item $G = \event{\text{тетраэдр упал на грань, содержащую желтый цвет}}$ ,
    \item $S = \event{\text{тетраэдр упал на грань, содержащую синий цвет}}$ .
\end{itemize}

Показать, что указанные события попарно независимы, но не являются независимыми в совокупности.

\subsection*{Решение:}

Множество элементарных исходов содержит четыре элемента $\Omega = \set{\omega_1, \omega_2, \omega_3, \omega_4}$. Каждый исход соответствует грани, на которую падает тетраэдр:
\begin{itemize}
    \item $\omega_1$ --- грань красного цвета,
    \item $\omega_2$ --- грань желтого цвета,
    \item $\omega_3$ --- грань синего цвета,
    \item $\omega_4$ --- грань, содержащая все три цвета.
\end{itemize}

События:
\begin{gather}
    K = \set{\omega_1, \omega_4} , \\
    G = \set{\omega_2, \omega_4} , \\
    S = \set{\omega_3, \omega_4} .
\end{gather}
их вероятности:
\begin{gather}
    \probability{K} = \frac{\modulus{K}}{\modulus{\Omega}} = \frac{2}{4} = \frac{1}{2} , \\
    \probability{G} = \frac{\modulus{G}}{\modulus{\Omega}} = \frac{2}{4} = \frac{1}{2} , \\
    \probability{S} = \frac{\modulus{S}}{\modulus{\Omega}} = \frac{2}{4} = \frac{1}{2} .
\end{gather}

Попарные произведения событий:
\begin{gather}
    KG = \set{\omega_4} , \\
    KS = \set{\omega_4} , \\
    GS = \set{\omega_4} .
\end{gather}
и их вероятности:
\begin{gather}
    \probability{KG} = \frac{\modulus{KG}}{\modulus{\Omega}} = \frac{1}{4} , \\
    \probability{KS} = \frac{\modulus{KS}}{\modulus{\Omega}} = \frac{1}{4} , \\
    \probability{GS} = \frac{\modulus{GS}}{\modulus{\Omega}} = \frac{1}{4} .
\end{gather}

Проверяем условия попарной независимости --- вероятность произведения должна быть равна произведению вероятностей:
\begin{gather}
    \frac{1}{4} = \probability{KG} = \probability{K} \cdot \probability{G} = \frac{1}{2} \cdot \frac{1}{2} , \\
    \frac{1}{4} = \probability{KS} = \probability{K} \cdot \probability{S} = \frac{1}{2} \cdot \frac{1}{2} , \\
    \frac{1}{4} = \probability{GS} = \probability{G} \cdot \probability{S} = \frac{1}{2} \cdot \frac{1}{2} .
\end{gather}
Равенства выполняются --- все события попарно независимы.

Событие произведения всех событий:
\begin{equation}
    KGS = \set{\omega_4}
\end{equation}
и его вероятность:
\begin{equation}
    \probability{KGS} = \frac{\modulus{KGS}}{\modulus{\Omega}} = \frac{1}{4} .
\end{equation}

Проверяем условие независимости в совокупности:
\begin{equation}
    \frac{1}{4} = \probability{KGS} \neq \probability{K} \cdot \probability{G} \cdot \probability{S} = \frac{1}{2} \cdot \frac{1}{2} \cdot \frac{1}{2} = \frac{1}{8} .
\end{equation}
Равенство не выполняется --- события не являются независимыми в совокупности.

% 167 179