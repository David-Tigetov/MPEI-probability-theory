\chapter{Вычисление вероятностей}

\section*{Введение}

Вычисление вероятности события --- одна из основных задач в теории вероятностей. Обычно, эта задача формулируется в следующем виде: заданы события с известными вероятностями,
требуется вычислить вероятность некоторого другого события, которое с ними связано, то есть выражается через события с известными вероятностями с помощью операций объединения,
произведения, дополнения, разности.

Решение задач подобного типа состоит из двух частей: в первой части необходимо получить выражение для события с неизвестной вероятностью, во второй части
вычислить вероятность с использованием известных соотношений, к которым относятся:
\begin{enumerate}
    \item вероятность дополнительного события:
    \begin{equation}
        \probability{\overline{A}} = 1 - \probability{A} ,
    \end{equation}

    \item формула сложения:
    \begin{equation}
        \probability{A + B} = \probability{A} + \probability{B} - \probability{A B} ,
    \end{equation}

    \item формула умножения:
    \begin{equation}
        \probability{A \cdot B} = \probability{A} \cdot \conditionalprobability{B}{A},
    \end{equation}

    \item формула полной вероятности:
    \begin{equation}
        \probability{B} = \sum_{i=1}^n \probability{A_i} \cdot \conditionalprobability{B}{A_i} ,
    \end{equation}
    где $A_1$, \dots, $A_n$ --- полная группа событий.

    \item формула Байеса:
    \begin{equation}
        \conditionalprobability{A_i}{B}
        = \frac{\probability{A_iB}}{\probability{B}}
        = \frac{\probability{A_i} \cdot \conditionalprobability{B}{A_i}}{\sum_{i=1}^n \probability{A_i} \cdot \conditionalprobability{B}{A_i}} ,
    \end{equation}
    где $A_1$, \dots, $A_n$ --- полная группа событий.
\end{enumerate}

