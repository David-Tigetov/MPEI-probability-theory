\documentclass[a4paper,12pt]{article}
\usepackage[T1]{fontenc}
\usepackage[utf8]{inputenc}
\usepackage[english,russian]{babel}
\usepackage[margin=2cm]{geometry}
\usepackage{amsmath}

\newcommand{\event}[1]{\left \{ #1 \right \}}
\newcommand{\set}[1]{\left \{ #1 \right \}}

\newcommand{\modulus}[1]{\left | #1 \right |}

\begin{document}

    \title{Практическое занятие 11. \\ Свойства математического ожидания и дисперсии.}
    \author{Тигетов Давид}
    \maketitle


    \section{Свойства}
    Пусть $X$ и $Y$ --- случайные величины, и $a$, $b$ и $c$ --- числа.

    Для математического ожидания справедливо свойство линейности:
    \begin{equation}
        \label{expectation}
        \expectation{a \cdot X + b \cdot Y + c}
        = a \cdot \expectation{X} + b \cdot \expectation{Y} + c
    \end{equation}

    Для дисперсии справедливо равенство:
    \begin{equation}
        \label{variance}
        \variance{a \cdot X + b \cdot Y + c}
        = a^2 \cdot \variance{X} + b^2 \cdot \variance{Y} + 2 a b \cdot \covariance{X}{Y}
    \end{equation}

    С использование свойств \eqref{expectation} и \eqref{variance} можно показать справедливость следующих выражений для дисперсии:
    \begin{multline}
        \variance{X}
        = \expectation{\left ( X - \expectation{X} \right )^2}
        = \expectation{X^2 - 2 X \cdot \expectation{X} + \expectation{X}^2} = \\
        %
        = \expectation{X^2} - 2 \expectation{X} \cdot \expectation{X} + \expectation{X}^2
        = \expectation{X^2} - \expectation{X}^2
    \end{multline}
    и ковариации:
    \begin{multline}
        \label{covariance}
        \covariance{X}{Y}
        = \expectation{\left ( X - \expectation{X} \right ) \left ( Y - \expectation{Y} \right )}
        = \expectation{ X Y - \expectation{X} Y - X \expectation{Y} + \expectation{X} \expectation{Y}} = \\
        %
        = \expectation{XY} - \expectation{X} \expectation{Y} - \expectation{X} \expectation{Y} + \expectation{X} \expectation{Y}
        = \expectation{XY} - \expectation{X} \expectation{Y} .
    \end{multline}


    \section{Задача 18.437}
    \section*{Условие}
    Случайные величины $X$ и $Y$ независимы и имеют следующие характеристики: $m_X = 1$, $m_Y = 2$, $\sigma_X = 1$, $\sigma_Y = 2$.
    Вычислить математические ожидания случайных величин:
    \begin{enumerate}
        \item $U = X^2 + 2 Y^2 - XY - 4X + Y + 4$,
        \item $V = (X + Y - 1)^2$
    \end{enumerate}
    \section*{Решение}
    \begin{enumerate}
        \item По свойству линейности математического ожидания:
        \begin{multline}
            \label{437:U}
            \expectation{U}
            = \expectation{X^2 + 2 Y^2 - XY - 4X + Y + 4} = \\
            = \expectation{X^2} + 2 \expectation{Y^2} - \expectation{XY} - 4 \expectation{X} + \expectation{Y} + \expectation{4}
        \end{multline}

        Для вычисления вторых начальных моментов используем известное соотношение для дисперсии:
        \begin{gather}
            \variance{X} = \expectation{X^2} - \left ( \expectation{X} \right )^2 , \\
            \variance{X} + \left ( \expectation{X} \right )^2 = \expectation{X^2} \label{437:X_2}.
        \end{gather}
        и
        \begin{equation}
            \label{437:Y_2}
            \expectation{Y^2} = \variance{Y} + \left ( \expectation{Y} \right )^2.
        \end{equation}

        По условию задачи величины $X$ и $Y$ \textbf{независимы}, поэтому:
        \begin{gather}
            \expectation{XY} = \expectation{X} \cdot \expectation{Y}
        \end{gather}

        Подставляя полученные выражения в равенство \eqref{437:U}, получим:
        \begin{multline}
            \expectation{U}
            = \variance{X} + \left ( \expectation{X} \right )^2 + 2 \left ( \variance{Y} + \left ( \expectation{Y} \right )^2 \right ) - \expectation{X} \cdot \expectation{Y} - 4 \expectation{X} + \expectation{Y} + \expectation{4} = \\
            = \sigma_X^2 + m_X^2 + 2 \left ( \sigma_Y^2 + m_Y^2 \right ) - m_X m_Y - 4 m_X + m_Y + 4 = \\
            = 1^2 + 1^2 + 2 \left ( 2^2 + 2^2 \right ) - 1 \cdot 2 - 4 \cdot 1 + 2 + 4 = 18
        \end{multline}

        \item Заметим, что величина $V$ является квадратом выражения, и требуется найти второй начальный момент:
        \begin{equation}
            \label{437:expectation}
            \expectation{V} = \expectation{\left ( X + Y - 1 \right )^2} = \variance{X + Y - 1} + \left ( \expectation{X + Y - 1} \right )^2 ,
        \end{equation}
        где в последнем равенстве использовалось выражение для второго момента через дисперсию и квадрат математического ожидания, как ранее, в выражениях \eqref{437:X_2}
        и \eqref{437:Y_2}.
        По свойствам дисперсии:
        \begin{equation}
            \label{437:variance}
            \variance{X + Y - 1} = \variance{X} + \variance{Y} + 2 \covariance{X}{Y}
        \end{equation}
        Ковариация величин $X$ и $Y$ из равенства \eqref{covariance}:
        \begin{equation}
            \covariance{X}{Y} = \expectation{XY} - \expectation{X} \expectation{Y}.
        \end{equation}
        В силу \textbf{независимости} величин $X$ и $Y$:
        \begin{equation}
            \expectation{XY} = \expectation{X} \expectation{Y} ,
        \end{equation}
        поэтому ковариация равна 0:
        \begin{equation}
            \covariance{X}{Y} = \expectation{X} \expectation{Y} - \expectation{X} \expectation{Y} = 0 ,
        \end{equation}
        а выражение для дисперсии \eqref{437:variance} принимает вид:
        \begin{equation}
            \variance{X + Y - 1} = \variance{X} + \variance{Y} .
        \end{equation}
        и выражение \eqref{437:expectation}:
        \begin{multline}
            \expectation{V}
            = \variance{X} + \variance{Y} + \left ( \expectation{X} + \expectation{Y} - 1 \right )^2 = \\
            %
            = \sigma_X^2 + \sigma_Y^2 + \left ( m_X + m_Y - 1 \right )^2
            = 1^2
            + 2^2 + \left ( 1 + 2 - 1 \right )^2
            = 9
        \end{multline}
    \end{enumerate}
    \section*{Ответ}
    \begin{enumerate}
        \item $\expectation{U} = 18$ ,
        \item $\expectation{V} = 9$.
    \end{enumerate}


    \section{Задача 18.438}
    \section*{Условие}
    Случайная точка $\left ( X, Y \right )$ характеризуется центром рассеивания $(-1, 1)$ и ковариационной матрицей
    $\begin{pmatrix}
         3  & -2 \\
         -2 & 4
    \end{pmatrix}$.
    Найти математическое ожидание и дисперсию случайной величины $Z = 2 X - 4 Y + 3$.

    \section*{Решение}
    Центр рассеивания:
    \begin{equation}
        \begin{pmatrix}
            -1 \\
            1
        \end{pmatrix}
        =
        \begin{pmatrix}
            \expectation{X} \\
            \expectation{Y}
        \end{pmatrix}
        .
    \end{equation}

    Ковариационная матрица
    \begin{equation}
        \begin{pmatrix}
            3  & -2 \\
            -2 & 4
        \end{pmatrix}
        =
        \begin{pmatrix}
            \variance{X}      & \covariance{X}{Y} \\
            \covariance{X}{Y} & \variance{Y}
        \end{pmatrix}
        .
    \end{equation}

    В силу линейности математического ожидания:
    \begin{equation}
        \expectation{Z}
        = \expectation{2X - 4 Y + 3}
        = 2 \expectation{X} - 4 \expectation{Y} + 3
        = 2 \cdot (-1) - 4 \cdot 1 + 3
        = -3
        .
    \end{equation}
    и по свойствам дисперсии:
    \begin{multline}
        \variance{Z}
        = \variance{2 X - 4 Y + 3}
        = \variance{2 X + (-4) Y + 3} = \\
        %
        = 2^2 \cdot \variance{X} + (-4)^2 \cdot \variance{Y} + 2 \cdot 2 \cdot (-4) \cdot \covariance{X}{Y} = \\
        %
        = 4 \cdot 3 + 16 \cdot 4 - 16 \cdot (-2)
        = 12 + 16 \cdot 6
        = 12 + 96
        = 108.
    \end{multline}

    \section*{Ответ}
    $\expectation{Z} = -3$, $\variance{Z} = 108$.


    \section{Задача 18.443}
    \section*{Условие}
    Случайная величина $X$ дискретного типа распределена по закону, определяемому таблицей:

    \begin{tabular}{|c|c|c|c|}
        \hline
        $x_i$ & -1            & 0             & 1             \\
        \hline
        $p_i$ & $\frac{1}{6}$ & $\frac{1}{3}$ & $\frac{1}{2}$ \\
        \hline
    \end{tabular}

    Найти коэффициент корреляции между $X$ и $X^2$.
    \section*{Решение}
    По определению коэффициент корреляции $\rho$:
    \begin{equation}
        \rho = \frac{\covariance{X}{X^2}}{\sqrt{\variance{X} \variance{X^2}}}
    \end{equation}

    Представим ковариацию с помощью моментов:
    \begin{equation}
        \covariance{X}{X^2}
        = \expectation{X \cdot X^2} - \expectation{X} \cdot \expectation{X^2}
        = \expectation{X^3} - \expectation{X} \cdot \expectation{X^2}
    \end{equation}

    Заметим, что у величины $X$ нечетные начальные моменты ($k=1,2,3,...$):
    \begin{equation}
        \expectation{X^{2k-1}} = \sum_{i=1}^3 x_i^{2k-1} \cdot p_i = (-1)^{2k-1} \cdot \frac{1}{6} + 0^{2k-1} \cdot \frac{1}{3} + 1^{2k-1} \cdot \frac{1}{2} = - \frac{1}{6} + \frac{1}{2} = \frac{1}{3} ,
    \end{equation}
    и чётные начальные моменты ($k=1,2,3,...$):
    \begin{equation}
        \expectation{X^{2k}} = \sum_{i=1}^3 x_i^{2k} \cdot p_i = (-1)^{2k} \cdot \frac{1}{6} + 0^{2k} \cdot \frac{1}{3} + 1^{2k} \cdot \frac{1}{2} = \frac{1}{6} + \frac{1}{2} = \frac{2}{3} .
    \end{equation}

    Таким образом, ковариация:
    \begin{equation}
        \covariance{X}{X^2} = \frac{1}{3} - \frac{1}{3} \cdot \frac{2}{3} = \frac{3}{9} - \frac{2}{9} = \frac{1}{9} .
    \end{equation}

    Вычислим дисперсии:
    \begin{gather}
        \variance{X} = \expectation{X^2} - \left ( \expectation{X} \right )^2 = \frac{2}{3} - \left ( \frac{1}{3} \right )^2 = \frac{6}{9} - \frac{1}{9} = \frac{5}{9} , \\
        \variance{X^2} = \expectation{X^4} - \left ( \expectation{X^2} \right )^2 = \frac{2}{3} - \left ( \frac{2}{3} \right )^2 = \frac{6}{9} - \frac{4}{9} = \frac{2}{9}.
    \end{gather}

    Таким образом, коэффициент корреляции:
    \begin{equation}
        \rho = \frac{\frac{1}{9}}{\sqrt{\frac{5}{9} \frac{2}{9}}} = \frac{\frac{1}{9}}{\frac{1}{9} \sqrt{10}} = \frac{1}{\sqrt{10}} .
    \end{equation}

    \section*{Ответ}
    $\frac{1}{\sqrt{10}}$


    \section{Домашнее задание}

    Задачи 436, 437, 439, 441, 444.

\end{document}