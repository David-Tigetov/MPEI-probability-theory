\chapter{Случайные величины непрерывного типа}


\section*{Задача 18.365}

В нормально распределенной совокупности 15\% значений $x$ меньше 12 и 40\% значений x больше 16.2. Найти среднее значение и стандартное отклонение данного распределения.
\subsection*{Решение}
В соответствии с условием имеем систему равенств:
\begin{gather}
    \left \{
    \begin{array}{lcl}
        \probability{x < 12}   & = & 0.15 \\
        \probability{x > 16.2} & = & 0.40
    \end{array}
    \right . \\
    %
    \left \{
    \begin{array}{lcl}
        \probability{x < 12}         & = & 0.15 \\
        1 - \probability{x \le 16.2} & = & 0.40
    \end{array}
    \right . \\
    %
    \left \{
    \begin{array}{lcl}
        \probability{x < 12}     & = & 0.15 \\
        \probability{x \le 16.2} & = & 0.60
    \end{array}
    \right .
\end{gather}
Пусть $F(x)$ --- функция распределения, тогда:
\begin{equation}
    \label{365:system}
    \left \{
    \begin{array}{lcl}
        F(12)   & = & 0.15 \\
        F(16.2) & = & 0.6
    \end{array}
    \right .
    .
\end{equation}
Пусть $m$ и $\sigma$ --- математическое ожидание и стандартное (среднеквадратическое) отклонение совокупности, которые необходимо найти, тогда:
\begin{equation}
    \label{365:F}
    F(x) = \Phi \left ( \frac{x - m}{\sigma} \right ),
\end{equation}
где $\Phi$ --- функция распределения стандартного нормального распределения $\mathcal{N}(0, 1)$. Используя равенство \eqref{365:F} в системе \eqref{365:system}, получим:
\begin{gather}
    \left \{
    \begin{array}{lcl}
        \Phi \left ( \frac{12-m}{\sigma} \right )   & = & 0.15 \\
        \Phi \left ( \frac{16.2-m}{\sigma} \right ) & = & 0.6
    \end{array}
    \right .
\end{gather}
Используя таблицы функции $\Phi(\cdot)$, получим значения в правой части равенств:
\begin{gather}
    \left \{
    \begin{array}{lcl}
        \frac{12-m}{\sigma}   & = & \Phi^{-1} \left (  0.15 \right ) \\
        \frac{16.2-m}{\sigma} & = & \Phi^{-1} \left ( 0.6 \right )
    \end{array}
    \right . \\
    %
    \left \{
    \begin{array}{lcl}
        \frac{12-m}{\sigma}   & \approx & -1.05 \\
        \frac{16.2-m}{\sigma} & \approx & 0.25
    \end{array}
    \right . \\
    %
    \left \{
    \begin{array}{lcl}
        12 - m   & \approx & -1.05 \sigma \\
        16.2 - m & \approx & 0.25 \sigma
    \end{array}
    \right . \\
    %
    \left \{
    \begin{array}{lcl}
        m - 1.05 \sigma & \approx & 12   \\
        m + 0.25 \sigma & \approx & 16.2
    \end{array}
    \right .
\end{gather}
Из второго уравнения вычитаем первое и складываем первое уравнения со вторым, умноженным на 4.2:
\begin{gather}
    \left \{
    \begin{array}{lcl}
        1.3 \sigma & \approx & 4.2                  \\
        5.2 m      & \approx & 16.2 \cdot 4.02 + 12
    \end{array}
    \right . \\
    %
    \left \{
    \begin{array}{lcl}
        \sigma & \approx & 3.23  \\
        m      & \approx & 15.39
    \end{array}
    \right .
\end{gather}

\subsection*{Ответ}
Математическое ожидание 3.23 и стандартное отклонение 15.39.
