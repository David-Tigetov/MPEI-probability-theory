\chapter{Случайные величины непрерывного типа}

\section*{Введение}



\section*{Задача 18.269}

Случайная величина $X$ распределена по закону, определяемому плотностью вероятностей вида:
\[
    f_X(x)
    = \left \{
    \begin{array}{ll}
        c \cos x , & \text{если } - \frac{\pi}{2} \le x \le \frac{\pi}{2} , \\
        0,         & \text{если } \modulus{x} > \frac{\pi}{2} .
    \end{array}
    \right .
\]
Необходимо:
\begin{enumerate}
    \item найти константу $c$,
    \item функцию распределения $F_X(x)$,
    \item вероятность $\probability{\modulus{X} < \frac{\pi}{4}}$,
    \item математическое ожидание $\expectation{X}$,
    \item дисперсию $\variance{X}$.
\end{enumerate}

\subsection*{Решение:}

Постоянная $c$ определяется из условия нормировки:
\begin{equation}
    \int \limits_{-\infty}^{\infty} f_X(x) dx = 1 .
\end{equation}
Подставляя выражение для плотности вероятности, получим уравнение:
\begin{gather}
    \int \limits_{-\frac{\pi}{2}}^{\frac{\pi}{2}} c \cos x dx = 1 , \\
    c \int \limits_{-\frac{\pi}{2}}^{\frac{\pi}{2}} \cos x dx = 1 , \\
    c \left . \sin x \right |_{-\frac{\pi}{2}}^{\frac{\pi}{2}} = 1 . \\
    c \cdot 2 = 1 , \\
    c = \frac{1}{2} .
\end{gather}

Функция распределения $F_X(x)$ связана с функцией плотности вероятности равенством:
\begin{gather}
    F_X(x) = \int \limits_{-\infty}^x f_X(t) dt .
\end{gather}
Учитывая носитель функции плотности распределения, получим:
\begin{equation}
    F_X(x)
    = \left \{
    \begin{array}{ll}
        0,                                              & \text{если } x < - \frac{\pi}{2},                    \\
        \int \limits_{-\infty}^x \frac{1}{2} \cos t dt, & \text{если } -\frac{\pi}{2} \le x \le \frac{\pi}{2}, \\
        1,                                              & \text{если } \frac{\pi}{2} < x .
    \end{array}
    \right .
\end{equation}
Вычисляя интеграл, получим выражение:
\begin{equation}
    F_X(x)
    = \left \{
    \begin{array}{ll}
        0,                                       & \text{если } x < - \frac{\pi}{2},                    \\
        \frac{1}{2} \left ( \sin x + 1 \right ), & \text{если } -\frac{\pi}{2} \le x \le \frac{\pi}{2}, \\
        1,                                       & \text{если } \frac{\pi}{2} < x .
    \end{array}
    \right .
\end{equation}

Вероятность события $\event{\modulus{X} < \frac{\pi}{4}}$ можно вычислить с помощью функции распределения или функции плотности вероятности:
\begin{gather}
    \probability{\modulus{X} < \frac{\pi}{4}}
    = \probability{-\frac{\pi}{4} < X < \frac{\pi}{4}}
    = \probability{X < \frac{\pi}{4}} - \probability{X \le -\frac{\pi}{4}} = \\
    = \left [
    \begin{array}{l}
        F_X \left ( \frac{\pi}{4} \right ) - \lim \limits_{x \rightarrow - \frac{\pi}{4}+0} F_X(x) , \\
        \int \limits_{-\frac{\pi}{4}}^{\frac{\pi}{4}} f_X(x) dx .
    \end{array}
    \right .
\end{gather}
С помощью функции распределения $F_X(x)$:
\begin{multline}
    \probability{\modulus{X} < \frac{\pi}{4}}
    = \frac{1}{2} \left ( \sin \frac{\pi}{4} + 1 \right ) - \frac{1}{2} \left ( \sin \left ( - \frac{\pi}{4} \right ) + 1 \right ) = \\
    %
    = \frac{1}{2} \sin \frac{\pi}{4} - \frac{1}{2} \sin \left ( - \frac{\pi}{4} \right )
    = \sin \frac{\pi}{4}
    = \frac{1}{\sqrt{2}} .
\end{multline}

Математическое ожидание:
\begin{equation}
    \expectation{X}
    = \int \limits_{-\infty}^{\infty} x \cdot f_X(x) dx
    = \int \limits_{-\frac{\pi}{2}}^{\frac{\pi}{4}} x \cdot \frac{1}{2} \cos x dx
    = 0,
\end{equation}
поскольку подынтегральная функция нечетная.

При вычислении дисперсии воспользуемся её представлением через начальные моменты:
\begin{equation}
    \variance{X} = \expectation{X^2} - \left ( \expectation{X} \right )^2 .
\end{equation}
Остаётся вычислить только второй начальный момент:
\begin{multline}
    \expectation{X^2}
    = \int \limits_{-\infty}^{\infty} x^2 \cdot f_X(x) dx
    = \int \limits_{-\frac{\pi}{2}}^{\frac{\pi}{4}} x^2 \cdot \frac{1}{2} \cos x dx = \\
    %
    = \frac{1}{2} \left (
    \left . x^2 \sin x \right |_{-\frac{\pi}{2}}^{\frac{\pi}{2}}
    - \left . 2 x \left ( - \cos x \right )\right |_{-\frac{\pi}{2}}^{\frac{\pi}{2}}
    + \left . 2 \left ( - \sin x \right )\right |_{-\frac{\pi}{2}}^{\frac{\pi}{2}}
    \right )
    = \\
    %
    = \frac{1}{2} \left (
    \left ( \frac{\pi}{2} \right )^2 +  \left ( \frac{\pi}{2} \right )^2
    - 0
    + 2 \cdot ( - 2 )
    \right )
    = \left ( \frac{\pi}{2} \right )^2 - 2.
\end{multline}
Дисперсия:
\begin{equation}
    \variance{X}
    = \left ( \frac{\pi}{2} \right )^2 - 2 - 0^2
    = \left ( \frac{\pi}{2} \right )^2 - 2.
\end{equation}

\subsection*{Ответ:}
\begin{enumerate}
    \item $c = \frac{1}{2}$,
    \item $F_X(x)
    = \left \{
    \begin{array}{ll}
        0,                                       & \text{если } x < - \frac{\pi}{2},                    \\
        \frac{1}{2} \left ( \sin x + 1 \right ), & \text{если } -\frac{\pi}{2} \le x \le \frac{\pi}{2}, \\
        1,                                       & \text{если } \frac{\pi}{2} < x .
    \end{array}
    \right .
    $
    \item $\probability{\modulus{X} < \frac{\pi}{4}} = \frac{1}{\sqrt{2}}$,
    \item $\expectation{X} = 0$,
    \item $\variance{X} = \left ( \frac{\pi}{2} \right )^2 - 2$.
\end{enumerate}

\section*{Задача 18.365}

В нормально распределенной совокупности 15\% значений $x$ меньше 12 и 40\% значений x больше 16.2. Найти среднее значение и стандартное отклонение данного распределения.
\subsection*{Решение}
В соответствии с условием имеем систему равенств:
\begin{gather}
    \left \{
    \begin{array}{lcl}
        \probability{x < 12}   & = & 0.15 \\
        \probability{x > 16.2} & = & 0.40
    \end{array}
    \right . \\
    %
    \left \{
    \begin{array}{lcl}
        \probability{x < 12}         & = & 0.15 \\
        1 - \probability{x \le 16.2} & = & 0.40
    \end{array}
    \right . \\
    %
    \left \{
    \begin{array}{lcl}
        \probability{x < 12}     & = & 0.15 \\
        \probability{x \le 16.2} & = & 0.60
    \end{array}
    \right .
\end{gather}
Пусть $F(x)$ --- функция распределения, тогда:
\begin{equation}
    \label{365:system}
    \left \{
    \begin{array}{lcl}
        F(12)   & = & 0.15 \\
        F(16.2) & = & 0.6
    \end{array}
    \right .
    .
\end{equation}
Пусть $m$ и $\sigma$ --- математическое ожидание и стандартное (среднеквадратическое) отклонение совокупности, которые необходимо найти, тогда:
\begin{equation}
    \label{365:F}
    F(x) = \Phi \left ( \frac{x - m}{\sigma} \right ),
\end{equation}
где $\Phi$ --- функция распределения стандартного нормального распределения $\mathcal{N}(0, 1)$. Используя равенство \eqref{365:F} в системе \eqref{365:system}, получим:
\begin{gather}
    \left \{
    \begin{array}{lcl}
        \Phi \left ( \frac{12-m}{\sigma} \right )   & = & 0.15 \\
        \Phi \left ( \frac{16.2-m}{\sigma} \right ) & = & 0.6
    \end{array}
    \right .
\end{gather}
Используя таблицы функции $\Phi(\cdot)$, получим значения в правой части равенств:
\begin{gather}
    \left \{
    \begin{array}{lcl}
        \frac{12-m}{\sigma}   & = & \Phi^{-1} \left (  0.15 \right ) \\
        \frac{16.2-m}{\sigma} & = & \Phi^{-1} \left ( 0.6 \right )
    \end{array}
    \right . \\
    %
    \left \{
    \begin{array}{lcl}
        \frac{12-m}{\sigma}   & \approx & -1.05 \\
        \frac{16.2-m}{\sigma} & \approx & 0.25
    \end{array}
    \right . \\
    %
    \left \{
    \begin{array}{lcl}
        12 - m   & \approx & -1.05 \sigma \\
        16.2 - m & \approx & 0.25 \sigma
    \end{array}
    \right . \\
    %
    \left \{
    \begin{array}{lcl}
        m - 1.05 \sigma & \approx & 12   \\
        m + 0.25 \sigma & \approx & 16.2
    \end{array}
    \right .
\end{gather}
Из второго уравнения вычитаем первое и складываем первое уравнения со вторым, умноженным на 4.2:
\begin{gather}
    \left \{
    \begin{array}{lcl}
        1.3 \sigma & \approx & 4.2                  \\
        5.2 m      & \approx & 16.2 \cdot 4.02 + 12
    \end{array}
    \right . \\
    %
    \left \{
    \begin{array}{lcl}
        \sigma & \approx & 3.23  \\
        m      & \approx & 15.39
    \end{array}
    \right .
\end{gather}

\subsection*{Ответ}
Математическое ожидание 3.23 и стандартное отклонение 15.39.
