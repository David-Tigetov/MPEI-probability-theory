\chapter{Классическое определение вероятности}

\section*{Введение}

В теории вероятностей решение задач выполняется в рамках вероятностного пространства $\left ( \Omega, \Sigma, P \right )$. В первом занятии разбирались вопросы формирования
множества всех элементарных исходов $\Omega$, событий из алгебры $\Sigma$ и операций над этими событиями. В этом занятии мы перейдем к одной из основных задач теории вероятностей
--- вычислению вероятностей с использованием заданной вероятностей меры $P$.

Вероятностная мера $P$ может быть задана различными способами. В частности, если множество элементарных исходов $\Omega$ имеет конечное число элементов $\modulus{\Omega} < \infty$,
то всем элементарным исходам $\omega_i \in \Omega$ ($i=1,\dots,\modulus{\Omega}$) можно приписать равные вероятности:
\begin{equation}
    \probability{\event{\omega_i}} = \frac{1}{\modulus{\Omega}}.
\end{equation}
Такое определение вероятностей меры называется \textbf{классическим определением вероятности}.

Теперь представим, что есть некоторое событие $A \in \Sigma$ из алгебры наблюдаемых событий. Как вычислить вероятность $\probability{A}$? Пусть событие $A$ состоит из $k$
элементарных исходов:
\begin{equation}
    A = \set{ \omega_{i_1}, \omega_{i_2}, \dots, \omega_{i_k}}.
\end{equation}
Мы знаем, что вероятностная мера $P$ обладает свойством аддитивности --- если два события $B$ и $C$ несовместны, то вероятности суммы событий есть сумма вероятностей:
\begin{equation}
    B \cap C = \emptyset : \; \probability{B + C} = \probability{B} + \probability{C}.
\end{equation}

Представим событие $A$ в виде объединения:
\begin{equation}
    A = \underbrace{\set{\omega_{i_1}}}_B \cup \underbrace{\set{\omega_{i_2}, \dots, \omega_{i_k}}}_C
\end{equation}
и воспользуемся свойством аддитивности меры $P$:
\begin{equation}
    \probability{A} = \probability{\set{\omega_{i_1}}} + \probability{\set{\omega_{i_2}, \dots, \omega_{i_k}}} .
\end{equation}
Поступая далее аналогичным образом, в итоге получим:
\begin{equation}
    \probability{A} = \probability{\set{\omega_{i_1}}} + \probability{\set{\omega_{i_2}}} + \dots + \probability{\set{\omega_{i_k}}} .
\end{equation}
Поскольку все элементарные исходы $\omega_{i_j}$ имеют одинаковые вероятности, то:
\begin{equation}
    \probability{A}
    = \underbrace{\frac{1}{\modulus{\Omega}} + \frac{1}{\modulus{\Omega}} + \dots + \frac{1}{\modulus{\Omega}}}_k
    = \frac{\modulus{A}}{\modulus{\Omega}} .
\end{equation}

Таким образом, вероятность определяется количествами элементарных исходов в событии $A$ и во всём множестве элементарных исходов $\Omega$.

\section*{Задача 18.66}

В магазин поступило 30 новых цветных телевизоров, среди которых 5 имеют скрытые дефекты. Наудачу отбирается один телевизор для проверки. Какова вероятность, что он не имеет
скрытых дефектов?

\subsection*{Решение:}

Пусть событие $A = \event{\text{выбранный телевизор не имеет дефектов}}$. Количество телевизоров, не имеющих скрытые дефекты, $30-5=25$, значит
\begin{equation}
    \modulus{A} = 25 .
\end{equation}
Всего телевизоров 30, значит
\begin{equation}
    \modulus{\Omega} = 30 .
\end{equation}
Таким образом, вероятность события $A$:
\begin{equation}
    \probability{A}
    = \frac{\modulus{A}}{\modulus{\Omega}}
    = \frac{25}{30}
    = \frac{5}{6}
    .
\end{equation}

\subsection*{Ответ:}
$\frac{5}{6}$.