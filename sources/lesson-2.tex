\chapter{Классическое определение вероятности}

\section*{Введение}

В теории вероятностей решение задач выполняется в рамках вероятностного пространства $\left ( \Omega, \Sigma, P \right )$. В первом занятии разбирались вопросы формирования
множества всех элементарных исходов $\Omega$, событий из алгебры $\Sigma$ и операций над этими событиями. В этом занятии мы перейдем к одной из основных задач теории вероятностей
--- вычислению вероятностей с использованием заданной вероятностей меры $P$.

Вероятностная мера $P$ может быть задана различными способами. В частности, если множество элементарных исходов $\Omega$ имеет конечное число элементов $\modulus{\Omega} < \infty$,
то всем элементарным исходам $\omega_i \in \Omega$ ($i=1,\dots,\modulus{\Omega}$) можно приписать равные вероятности:
\begin{equation}
    \probability{\event{\omega_i}} = \frac{1}{\modulus{\Omega}}.
\end{equation}
Такое определение вероятностей меры называется \textbf{классическим определением вероятности}.

Теперь представим, что есть некоторое событие $A \in \Sigma$ из алгебры наблюдаемых событий. Как вычислить вероятность $\probability{A}$? Пусть событие $A$ состоит из $k$
элементарных исходов:
\begin{equation}
    A = \set{ \omega_{i_1}, \omega_{i_2}, \dots, \omega_{i_k}}.
\end{equation}
Мы знаем, что вероятностная мера $P$ обладает свойством аддитивности --- если два события $B$ и $C$ несовместны, то вероятности суммы событий есть сумма вероятностей:
\begin{equation}
    B \cap C = \emptyset : \; \probability{B + C} = \probability{B} + \probability{C}.
\end{equation}

Представим событие $A$ в виде объединения:
\begin{equation}
    A = \underbrace{\set{\omega_{i_1}}}_B \cup \underbrace{\set{\omega_{i_2}, \dots, \omega_{i_k}}}_C
\end{equation}
и воспользуемся свойством аддитивности меры $P$:
\begin{equation}
    \probability{A} = \probability{\set{\omega_{i_1}}} + \probability{\set{\omega_{i_2}, \dots, \omega_{i_k}}} .
\end{equation}
Поступая далее аналогичным образом, в итоге получим:
\begin{equation}
    \probability{A} = \probability{\set{\omega_{i_1}}} + \probability{\set{\omega_{i_2}}} + \dots + \probability{\set{\omega_{i_k}}} .
\end{equation}
Поскольку все элементарные исходы $\omega_{i_j}$ имеют одинаковые вероятности, то:
\begin{equation}
    \probability{A}
    = \underbrace{\frac{1}{\modulus{\Omega}} + \frac{1}{\modulus{\Omega}} + \dots + \frac{1}{\modulus{\Omega}}}_k
    = \frac{\modulus{A}}{\modulus{\Omega}} .
\end{equation}

Таким образом, вероятность определяется количествами элементарных исходов в событии $A$ и во всём множестве элементарных исходов $\Omega$.

\section*{Задача 18.66}

В магазин поступило 30 новых цветных телевизоров, среди которых 5 имеют скрытые дефекты. Наудачу отбирается один телевизор для проверки. Какова вероятность, что он не имеет
скрытых дефектов?

\subsection*{Решение:}

Пусть событие $A = \event{\text{выбранный телевизор не имеет дефектов}}$. Количество телевизоров, не имеющих скрытые дефекты, $30-5=25$, значит
\begin{equation}
    \modulus{A} = 25 .
\end{equation}
Всего телевизоров 30, значит
\begin{equation}
    \modulus{\Omega} = 30 .
\end{equation}
Таким образом, вероятность события $A$:
\begin{equation}
    \probability{A}
    = \frac{\modulus{A}}{\modulus{\Omega}}
    = \frac{25}{30}
    = \frac{5}{6}
    .
\end{equation}

\subsection*{Ответ:}
$\frac{5}{6}$.

\section*{Задача [18.70]}

Подбрасываются два кубика. Фиксируются количества очков, выпавших на первом и втором кубиках. Найти вероятности событий
\begin{enumerate}
    \item $A = \event{\text{сумма очков нечётна}}$,
    \item $B = \event{\text{произведение очков чётно}}$.
\end{enumerate}

\subsection*{Решение:}

\begin{enumerate}
    \item
    Множество элементарных исходов $\Omega = \set{(i, j): i, j \in \left \{ 1, 2, 3, 4, 5, 6\right \}}$ содержит 36 элементарных исходов:
    \begin{equation}
        \modulus{\Omega} = 6 \cdot 6 = 36.
    \end{equation}

    \item
    Для подсчёта количества элементарных исходов в событии $A$ составим таблицу суммы:
    \begin{center}
        \begin{tabular}{c|c|c|c|c|c|c|}
            $i$, $j$ & 1 & 2 & 3 & 4  & 5  & 6  \\
            \hline
            1        & 2 & 3 & 4 & 5  & 6  & 7  \\
            \hline
            2        & 3 & 4 & 5 & 6  & 7  & 8  \\
            \hline
            3        & 4 & 5 & 6 & 7  & 8  & 9  \\
            \hline
            4        & 5 & 6 & 7 & 8  & 9  & 10 \\
            \hline
            5        & 6 & 7 & 8 & 9  & 10 & 11 \\
            \hline
            6        & 7 & 8 & 9 & 10 & 11 & 12 \\
            \hline
        \end{tabular}
    \end{center}
    Теперь подсчитаем количество клеток с нечетными числами (по диагоналям):
    \begin{equation}
        \modulus{A} = 2 + 4 + 6 + 4 + 2 = 18 .
    \end{equation}
    Таким образом, вероятность
    \begin{equation}
        \probability{A} = \frac{\modulus{A}}{\modulus{\Omega}} = \frac{18}{36} = \frac{1}{2} .
    \end{equation}

    При подсчёте количества элементарных исходов в событии $A$ можно было рассуждать и следующим образом: для получения нечётной суммы достаточно чтобы
    \begin{itemize}
        \item на первом кубике выпало нечётное число, а на втором --- чётное,
        \item на первом кубике выпало чётное число, а на втором --- нечётное.
    \end{itemize}
    Поскольку нечётных очков $\set{1, 3, 5}$ всего 3, и четных $\set{2, 4, 6}$ тоже 3, то
    \begin{equation}
        \modulus{A} = 3 \cdot 3 + 3 \cdot 3 = 18.
    \end{equation}

    \item
    Для подсчёта количества элементарных исходов в событии $B$ составим таблицу произведения:
    \begin{center}
        \begin{tabular}{c|c|c|c|c|c|c|}
            $i$, $j$ & 1 & 2  & 3  & 4  & 5  & 6  \\
            \hline
            1        & 1 & 2  & 3  & 4  & 5  & 6  \\
            \hline
            2        & 2 & 4  & 6  & 8  & 10 & 12 \\
            \hline
            3        & 3 & 6  & 9  & 12 & 15 & 18 \\
            \hline
            4        & 4 & 8  & 12 & 16 & 20 & 24 \\
            \hline
            5        & 5 & 10 & 15 & 20 & 25 & 30 \\
            \hline
            6        & 6 & 12 & 18 & 24 & 30 & 36 \\
            \hline
        \end{tabular}
    \end{center}
    Подсчёт по строкам количества клеток с чётным числом приводит результату:
    \begin{equation}
        \modulus{B} = 3 + 6 + 3 + 6 + 3 + 6 = 27 .
    \end{equation}
    Таким образом, вероятность
    \begin{equation}
        \probability{B} = \frac{\modulus{B}}{\modulus{\Omega}} = \frac{27}{36} = \frac{3}{4} .
    \end{equation}

    Количество элементарных исходов в $B$ можно было подсчитать и следующим образом: для получения чётного произведения достаточно чтобы
    \begin{itemize}
        \item на первом кубике выпало нечётное число, а на втором --- чётное,
        \item на первом кубике выпало чётное число, а на втором --- любое.
    \end{itemize}
    Таким образом,
    \begin{equation}
        \modulus{B} = 3 \cdot 3 + 3 \cdot 6 = 9 + 18 = 27 .
    \end{equation}
\end{enumerate}

\section*{Ответ:}

$\probability{A} = \frac{1}{2}$, $\probability{B} = \frac{3}{4}$.