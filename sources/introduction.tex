\chapter*{Введение}


\section*{Литература}

Теория:
\begin{enumerate}
    \item Гнеденко Б.В. "Курс теории вероятностей"{},
    \item Севастьянов Б.А. "Курс теории вероятностей"{},
    \item Горицкий Ю.А. "Введение в теорию вероятностей"{}.
\end{enumerate}

Практика:
\begin{enumerate}
    \item Ефимов А.В., Поспелов А.С. "Сборник задач по математике. Том 4"{},
    \item Чудесенко В.Ф. "Сборник заданий по специальным курсам высшей математики (типовые расчеты). Часть II."{}.
\end{enumerate}


\section*{Занятия}

Занятия состоят из:
\begin{enumerate}
    \item текста с решениями задач;
    \item видео-урока с пояснениями к решению задач;
    \item набора задач для самостоятельного решения:
    \begin{enumerate}
        \item на занятии,
        \item дома.
    \end{enumerate}
\end{enumerate}


\section*{Сдача}

Решения задач необходимо присылать по электронной почте:
\begin{enumerate}
    \item адреса: david.tigetov@gmail.com, TigetovDG@mpei.ru;
    \item тема письма: Группа Фамилия Имя [ Задачи | Типовой расчет | Контрольное мероприятие {1,2,3,4}]
\end{enumerate}

В решении задач обязательно должны быть:
\begin{enumerate}
    \item номер задачи (условие можно не переписывать),
    \item пояснения к решению,
    \item ответ,
    \item номера страниц.
\end{enumerate}