\chapter{Распределения Бернулли и Пуассона.}

\section*{Введение}

Случайная величина $\xi(\omega)$ --- это числовая функция, определенная на множестве элементарных исходов:
\begin{equation}
    \xi : \Omega \rightarrow \mathbb{R}.
\end{equation}

Примем краткое обозначение для $A \subseteq \mathbb{R}$:
\begin{equation}
    \xi \in A = \event{\omega : \xi(\omega) \in A}.
\end{equation}

Исчерпывающей характеристикой случайной величины является функция распределения:
\begin{equation}
    F_\xi(x) = \probability{\xi < x}.
\end{equation}

Случайные величины подразделяются на типы в зависимости от множества значений. Случайные величины дискретного типа имеют конечное или счетное множество значений.

Если случайная величина дискретного типа принимает значения $x_i$ с вероятностями $p_i$, то закон распределения такой случайной величины удобно представлять в виде таблицы:
\begin{center}
    \begin{tabular}{|c|c|c|c|c|}
        \hline
        $x_1$ & $x_2$ & \dots & $x_k$ & \dots \\
        \hline
        $p_1$ & $p_2$ & \dots & $p_k$ & \dots \\
        \hline
    \end{tabular}
\end{center}

Основные числовые характеристики случайной величины $\xi$:
\begin{itemize}
    \item математическое ожидание
    \begin{equation}
        \expectation{\xi} = \sum_{i=1}^{\infty} x_i p_i ,
    \end{equation}
    \item дисперсия
    \begin{equation}
        \variance{\xi} = \sum_{i=1}^{\infty} \left ( x_i - \expectation{\xi} \right )^2 p_i ,
    \end{equation}
    \item среднеквадратическое отклонение
    \begin{equation}
        \sigma_\xi = \sqrt{\variance{\xi}} .
    \end{equation}
\end{itemize}

Схема Бернулли: $n$ независимых испытаний, в каждом из которых с вероятностью $p$ можно происходить событие "успех"{}.
Случайная величина $\xi$ количества успехов имеет распределение Бернулли (биномиальное распределение) $\xi \sim B(n,p)$:
\begin{gather}
    \xi : \Omega \rightarrow \set{0, 1, \dots, n} \notag, \\
    \probability{\xi = k} = C_n^k p^k (1-p)^{n-k} .
\end{gather}

Случайная величина $\xi$ имеет распределение Пуассона $\xi \sim P(\lambda)$:
\begin{gather}
    \xi : \Omega \rightarrow \mathbb{N}, \\
    \probability{\xi = k} = \frac{\lambda^k}{k!} e^{-\lambda} .
\end{gather}

