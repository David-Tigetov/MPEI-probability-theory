\documentclass[12pt,a4paper]{article}

\usepackage[T1]{fontenc}
\usepackage[utf8]{inputenc}
\usepackage[english,russian]{babel}
\usepackage[margin=2cm]{geometry}

\begin{document}
    {
        \centering
        \textbf{Программа зачёта}
        \par
    }

    \begin{enumerate}
        \item Основные понятия: случайное событие, вероятность. Отношение событий. Вероятностное пространство.
        \item Условная вероятность, формула умножения вероятностей, независимость случайных событий.
        \item Формула полной вероятности и формула Байеса.
        \item Классическое определение вероятности. Геометрические вероятности. Задача о встрече.
        \item Функции распределения и их свойства. Дискретные и непрерывные случайные величины.
        \item Независимость случайных величин. Условные распределения.
        \item Одномерные случайные величины. Независимые испытания Бернулли.
        \item Однородный пуассоновский поток случайных точек.
        \item Теорема Муавра -- Лапласа. Теорема Пуассона.
        \item Числовые характеристики случайных величин: математическое ожидание, дисперсия. Примеры: распределение Бернулли, Пуассона, нормальное, равномерное.
        \item Интеграл Стильтьеса. Общее определение математического ожидания. Свойства математического ожидания и дисперсии. Примеры.
        \item Преобразование случайных величин. Примеры: линейное преобразование, возведение в квадрат, логарифмически нормальное распределение.
        \item Математическое ожидание функции от случайной величины. Моменты случайной величины (моменты распределения).
        \item Многомерные случайные величины: дискретные и непрерывные, функции распределения и их свойства.
        \item Числовые характеристики многомерных случайных величин. Коэффициент корреляции и его свойства.
        \item Преобразование многомерных случайных величин. Распределение суммы двух случайных величин, суммы квадратов независимых нормальных.
        \item Центральная предельная теорема. Пример применения в задаче оценки ошибок округления.
        \item Неравенство Чебышева. Закон больших чисел в форме Чебышева.
    \end{enumerate}
\end{document}
